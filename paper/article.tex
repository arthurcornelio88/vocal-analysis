\documentclass[review,3p,number]{elsarticle}

\usepackage[utf8]{inputenc}
\usepackage[T1]{fontenc}
\usepackage{amsmath}
\usepackage{amssymb}
\usepackage{graphicx}
\usepackage{booktabs}
\usepackage{siunitx}
\usepackage{xurl}
\usepackage{hyperref}
\usepackage{xcolor}
\usepackage{multirow}
\usepackage{lineno}

\linenumbers

\journal{Journal of Voice}

\begin{document}

\begin{frontmatter}

\title{Multidimensional Vocological Analysis: Laryngeal Mechanism Physiology, Digital Signal Processing, and a Critique of the Fach System in Brazilian Choro}

\author[ind]{Arthur Cornélio\corref{cor1}}
\ead{arthur.cornelio@gmail.com}
\cortext[cor1]{Corresponding author}

\affiliation[ind]{organization={Independent Researcher},
  city={Paris},
  country={France}}

\begin{abstract}
Modern vocology has evolved from a subjective discipline into a rigorous science
that integrates laryngeal physiology, acoustics, and signal processing. However,
the traditional German \emph{Fach} system---developed for operatic voice
classification---remains widely applied to genres for which it was never designed.
This study presents a computational bioacoustic analysis of laryngeal mechanisms
(M1/M2) in historical recordings of Brazilian Choro singing, using Ademilde Fonseca
as a case study.

We developed an open-source hybrid pipeline combining source separation (HTDemucs),
neural pitch tracking (CREPE), spectral feature extraction (Praat/Parselmouth), and
machine learning classification (Gaussian Mixture Model and XGBoost). We introduce
the Vocal Mechanism Index (VMI), a continuous spectral metric (0--1) that classifies
laryngeal mechanisms independently of tessitura, based on Alpha Ratio, $H_1$-$H_2$,
Spectral Tilt, and Cepstral Peak Prominence.

Analysis of 20,266 voiced frames across three recordings revealed that 57.3\% of
frames operated in M1 (mean $f_0$ = 315.9~Hz, D\#4) and 42.7\% in M2 (mean $f_0$ =
502.1~Hz, B4). The VMI distribution showed 44.3\% of frames in the passaggio zone
(VMI 0.4--0.6), indicating fluid register transitions. XGBoost classification
achieved 1.00 precision, recall, and F1-score on a held-out test set of 4,054
frames, confirming clear cluster separability. Formant analysis revealed speech-like
articulation ($F_1 \approx$ 660~Hz, $F_2 \approx$ 1,629~Hz), consistent with the
colloquial expressivity characteristic of Choro aesthetics.

These findings demonstrate that Ademilde Fonseca's vocal technique integrates both
laryngeal mechanisms fluidly, prioritizing textual intelligibility and rhythmic
agility over acoustic projection. The \emph{Fach} system fails to capture this
biomechanical reality, supporting the need for genre-specific, physiologically
grounded vocal classification frameworks.
\end{abstract}

\begin{keyword}
Laryngeal mechanisms \sep Fach system \sep Digital signal processing \sep
Choro \sep Vocal Mechanism Index \sep XGBoost
\end{keyword}

\end{frontmatter}

%% ============================================================
\section{Introduction}
\label{sec:introduction}
%% ============================================================

The scientific study of the singing voice has undergone a fundamental paradigm shift
over the past three decades. What was once an auxiliary discipline dependent on
subjective auditory evaluation has become a rigorous interdisciplinary science at the
intersection of laryngeal physiology, acoustics, ethnomusicology, and signal
engineering.\cite{roubeau2009,henrich2006} Modern technologies---including
electroglottography (EGG), high-speed laryngoscopy, and deep learning
algorithms---now enable the quantification of biomechanical phenomena that were
previously described only through perceptual metaphors.

Despite these advances, the classification of singing voices in Western pedagogy
remains dominated by the German \emph{Fach} system, a taxonomic framework developed
in the 19th century for European opera.\cite{cotton2007,miller2000} The \emph{Fach}
system categorizes voices based on tessitura, timbre, and the acoustic demands of
operatic repertoire. While effective within its original context, this classification
framework presents fundamental problems when applied to non-operatic genres,
particularly those from non-Western musical traditions.

Brazilian Choro---a genre that emerged in the late 19th century and constitutes one
of the foundations of Brazilian popular music---presents a particularly compelling
case for this critique. Choro singing demands specific vocal competencies that
diverge substantially from operatic ideals: rapid ornamental passages (glissandi,
portamenti), precise diction at high speed, rhythmic agility, and what Tatit
terms \emph{figurativiza\c{c}\~{a}o}---the speech-like quality of Brazilian
popular singing that privileges natural articulation over resonant
projection.\cite{tatit2002,rezende2016}

The central problem addressed in this study is the inadequacy of the \emph{Fach}
system for describing the vocal biomechanics of Choro singers. When applied to this
genre, the \emph{Fach} framework imposes rigid tessitura boundaries, fixed
\emph{passaggio} zones, and acoustic projection requirements that are aesthetically
irrelevant and pedagogically counterproductive in a microphone-dependent genre. The
essential vocal competencies of Choro---vocal attack precision, articulatory
diction at high speed, ornamental agility, register management, resonance
adjustments, and controlled vibrato\cite{rezende2016}---cannot be captured by a system designed to categorize
operatic voices by their capacity to fill a concert hall without amplification.

Recent advances in machine learning have opened new possibilities for objective
vocal analysis. XGBoost algorithms optimized by Differential Evolution have achieved
vocal register classification accuracy exceeding 97\% in controlled
settings.\cite{boratto2025} Hybrid methods combining Deep Neural Networks and
Linear Prediction (LPC) enable robust formant tracking even in high-pitched
voices.\cite{alku2023,gowda2022} These tools, combined with neural source separation
models capable of isolating vocal lines from complex arrangements,\cite{defossez2021,rouard2023}
now enable rigorous biomechanical analysis of singing in genres that were previously
inaccessible to computational methods.

To address this gap, we propose a computational framework based on explainable
bioacoustic features. Using digital signal processing (DSP) and machine learning, we
analyze historical recordings of Ademilde Fonseca (1921--2012)---widely recognized as
the ``Queen of Choro'' and one of the most influential interpreters in the genre's
history---to provide quantitative evidence that: (1) the singer employs both
laryngeal mechanisms (M1 and M2) fluidly across her repertoire; (2) her vocal
technique prioritizes colloquial expressivity over the acoustic projection criteria
central to \emph{Fach} classification; and (3) a continuous spectral metric (the
Vocal Mechanism Index, VMI) provides a more accurate and portable characterization
of laryngeal behavior than categorical \emph{Fach} labels.

This study fills a gap in the literature by providing the first computational
bioacoustic analysis of Brazilian Choro singing. While recent work has applied
machine learning to vocal register classification in contemporary pop
music,\cite{kim2025} controlled singing experiments,\cite{boratto2025} and singing voice
mode classification,\cite{hinrichs2026} no
previous study has addressed the specific acoustic challenges of historical Choro
recordings or the theoretical implications for \emph{Fach} critique in vernacular
genres. The novelty of this research lies in the application of a hybrid pipeline
(HTDemucs, CREPE, XGBoost) specifically designed to overcome the challenges of noisy
historical recordings, combined with the introduction of the VMI as a
tessitura-agnostic classification metric.

%% ============================================================
\section{Theoretical Framework}
\label{sec:framework}
%% ============================================================

\subsection{Critique of the Fach System}
\label{sec:fach_critique}

The \emph{Fach} system's central premise is vocal audibility over a non-amplified
orchestra. This capacity depends on the ``singer's formant cluster''---a peak of
acoustic energy between 2.5 and 3.5~kHz generated by the convergence of the third,
fourth, and fifth vocal tract formants ($F_3$, $F_4$, $F_5$).\cite{sundberg1974} This
resonance strategy allows operatic singers to project above an orchestral
accompaniment without electronic amplification.

In Choro and Brazilian popular music (MPB), however, the ubiquitous use of
microphones renders this acoustic requirement obsolete and, indeed, aesthetically
undesirable. Amplification allows low-intensity vocal qualities with speech-like
spectra, where energy naturally decays in high frequencies. The ``brightness'' and
``projection'' that define operatic \emph{Fach} categories are neither sought nor
valued. Classifying a Choro voice by its natural acoustic ``weight'' ignores the
electroacoustic reality inherent to the genre.\cite{bourne2012}

Beyond the acoustic projection fallacy, the \emph{Fach} system imposes rigid
tessitura limits and fixed \emph{passaggio} zones. In practice, popular music
singers operate in a flexible ``mixing zone,'' using vocal tract adjustments to
produce chest voice ``belting'' or to bring head voice qualities to lower registers,
subverting the fixed transitions assumed by \emph{Fach}
classification.\cite{bourne2012} This flexibility is particularly pronounced in
Choro, where singers must navigate rapid ornamental passages that cross register
boundaries within a single phrase.

Forcing Choro singers into \emph{Fach} categories leads to reductive
``pigeonholing'' that ignores individual ``comfort tessitura''---the range within
which a singer achieves optimal articulatory agility and textual
intelligibility.\cite{rezende2016} The concept of comfort tessitura is central to
Choro vocal technique: rather than striving to extend the range in pursuit of a
\emph{Fach}-appropriate tessitura, the Choro singer works within the region where
articulatory precision is maximized and where the vocal mechanism can support the
rapid ornamentations---mordents, turns, glissandi, and portamenti---that define the
genre's aesthetic identity.

Cotton's doctoral research\cite{cotton2007} examined the historical fluidity and
conflicting criteria of the \emph{Fach} system, demonstrating that even within its
native operatic context, voice classification criteria remain contested among
pedagogues and lack standardized application. When extended to genres with fundamentally different acoustic goals, these
inconsistencies are compounded by the system's irrelevant premises. The inadequacy of
\emph{Fach} in describing popular vocal biomechanics demands a more objective
paradigm: one based on quantifiable glottic behaviors rather than auditory metaphors.

\subsection{Laryngeal Mechanism Theory (M0--M3)}
\label{sec:mechanisms}

The scientific response to this classification gap lies in the theory of laryngeal
vibration mechanisms, established by Roubeau, Henrich, and
Castellengo.\cite{roubeau2009} This framework replaces the ambiguous concept of
``vocal registers'' with a physiological-quantitative taxonomy based on directly
measurable glottic parameters: vibratory mass, closed quotient, and neuromuscular
activation patterns.

Four mechanisms are defined:

\begin{description}
  \item[M0 (vocal fry):] Extremely low fundamental frequency, with irregular vocal
    fold vibration and minimal subglottic pressure. An expressive resource vital in
    popular music but marginalized in traditional \emph{bel canto}.
  \item[M1 (chest/modal voice):] Dominated by thyroarytenoid muscle contraction,
    producing thick vocal fold vibration with high closed quotient. M1 generates
    rich harmonic spectra with high spectral energy in low harmonics, increased
    harmonic richness in high frequencies, and strong acoustic
    projection.\cite{henrich2004}
  \item[M2 (head voice/falsetto):] Operates under cricothyroid muscle tension with
    reduced vibratory mass. M2 produces lower overall amplitude, energy concentrated
    in high harmonics, and characteristic phase transitions at register
    boundaries.\cite{roubeau2009}
  \item[M3 (whistle):] At the extreme high end of tessitura, similar to M0 in
    expressivity value within popular genres.
\end{description}

Critically, the M1/M2 overlap zone---spanning approximately one octave in vocal
extension---is where it is physiologically possible to phonate in either mechanism.
The choice within this zone is determined not solely by fundamental frequency ($f_0$)
but by desired intensity and timbre.\cite{roubeau2009,henrich2004} Henrich et
al.\cite{henrich2004} demonstrated through EGG measurements that the glottal open
quotient (OQ) provides the most reliable differentiation between mechanisms: M1
exhibits lower OQ (longer closed phase, more complete glottal contact), while M2
shows higher OQ (shorter closed phase, reduced contact area). These physiological
differences manifest acoustically as distinct spectral signatures---the basis for
the features used in our computational pipeline.

In Choro, technical mastery resides in the ability to transit or maintain mechanisms
within the overlap zone to serve the musical narrative: M1 for intensity and rhythmic
drive, M2 for ornamentation and phrase endings. Trained singers employ sophisticated
vocal tract adjustments at the \emph{passaggio}---for instance, opening the jaw
wider to raise $F_1$ and maintain it above $f_0$ (\emph{vowel tuning} or \emph{formant
tuning})---to achieve seamless transitions between mechanisms without perceptible
breaks.\cite{bozeman2013,sundberg1987} This biomechanical flexibility is precisely
what the \emph{Fach} system's rigid categories fail to accommodate.

%% ============================================================
\section{Materials and Methods}
\label{sec:methods}
%% ============================================================

\subsection{Audio Corpus}
\label{sec:corpus}

We analyzed three historical recordings of Ademilde Fonseca (1921--2012), one of the
most prominent Choro singers in Brazilian music history: \emph{Apanhei-te Cavaquinho}
(Ernesto Nazareth, arr. Altamiro Carrilho), \emph{Delicado} (Waldir Azevedo), and
\emph{Brasileirinho} (Waldir Azevedo). These recordings date from the 1940s--1960s
and present characteristic challenges for acoustic analysis: elevated background
noise, low signal-to-noise ratio (SNR), dense instrumental arrangements (7-string
guitar, cavaquinho, pandeiro, flute), and degraded spectral quality due to the
recording technology of the era.

The selection criteria prioritized: (1) musical relevance (canonical Choro
repertoire); (2) mechanism variability (passages with known M1/M2 alternation); and
(3) recording diversity (different accompaniment densities and tempi). These three
recordings encompass a range of musical demands---from the moderate \emph{Delicado}
to the virtuosic \emph{Brasileirinho}---providing a representative cross-section of
Fonseca's vocal technique.

Audio preprocessing followed a standardized protocol: all recordings were normalized
to $-3$~dBFS peak amplitude using the formula $x_{\text{norm}} = x \cdot
(10^{-3/20} / \max|x|)$, converted to mono (human voice being a point source,
stereo information is unnecessary), and resampled to 44.1~kHz (CD quality standard,
supporting frequency analysis up to 22.05~kHz). Non-vocal frames were filtered by
CREPE periodicity (confidence $> 0.8$) and HNR ($> -10$~dB), yielding 20,266 voiced
frames for analysis.

\subsection{Source Separation}
\label{sec:separation}

In complex Choro arrangements, pitch detection algorithms may capture instrumental
frequencies instead of the voice. We applied HTDemucs\cite{rouard2023}---a hybrid
source separation model combining temporal and spectral processing with Transformer
layers---to isolate the vocal line prior to feature extraction. HTDemucs achieves
$\approx 9$~dB Signal-to-Distortion Ratio (SDR) on source separation benchmarks,
substantially outperforming earlier methods such as Spleeter (5--6~dB SDR).

Separation results were validated by comparing $f_0$ contours extracted from the
original mix and the isolated vocal track (Figure~\ref{fig:separation}). Successful
separation was confirmed by: (1) elimination of spurious pitch detections in the
instrumental range ($< 200$~Hz), which were caused by the 7-string guitar bass lines
and cavaquinho harmonics; (2) a more continuous melodic contour in the expected
female voice range (200--600~Hz); and (3) increased average HNR (from $\approx 5$~dB
in the original mix to $\approx 17$~dB after separation), indicating that the
previously low HNR was attributable to instrumental interference rather than
physiological register differences. Separated audio was cached in NumPy format to
avoid reprocessing, with each song's separated vocal track available for independent
verification.

\subsection{Feature Extraction}
\label{sec:features}

The pipeline combines two complementary extraction methods, all operating at 5~ms
temporal resolution (hop length = 220 samples at 44.1~kHz):

\paragraph{Fundamental frequency ($f_0$).}
We used CREPE,\cite{kim2018} a convolutional neural network trained on annotated
pitch data, configured with the \texttt{full} model for maximum precision, frequency
range 50--800~Hz, and \texttt{weighted\_argmax} decoding. CREPE was selected over
autocorrelation-based methods (e.g., Praat) due to its robustness to intense
vibrato, background noise, and rapid ornamentations characteristic of historical
Choro recordings.

\paragraph{Vocal quality features.}
Using Praat via Parselmouth,\cite{boersma2023} we extracted: Harmonicity-to-Noise
Ratio (HNR), Cepstral Peak Prominence Smoothed (CPPS),\cite{maryn2015} Jitter
(ppq5), and Shimmer (apq11).\cite{teixeira2013,boersma2023} Formants $F_1$--$F_4$ were extracted via
Burg Linear Predictive Coding (5 formants, maximum formant 5,500~Hz). RMS spectral
energy was computed using librosa.

\paragraph{Spectral features for VMI.}
Four spectral features, each with established physiological correlates, were
extracted for the Vocal Mechanism Index:

\emph{Alpha Ratio}\cite{sundberg1987,yousef2024}: the energy ratio (dB) between the high band
(1--5~kHz) and the low band (50~Hz--1~kHz). In M1, rapid and firm glottal closure
generates strong upper harmonics, producing a high Alpha Ratio. In M2/falsetto,
gentle closure concentrates energy in the fundamental, yielding a low ratio.

\emph{$H_1$-$H_2$}\cite{kreiman2012,kreiman2014}: the amplitude difference (dB) between
the first harmonic ($H_1 = f_0$) and the second harmonic ($H_2 = 2f_0$), a well-established
acoustic correlate of the glottal adduction pattern. Firm adduction (M1) produces
abrupt airflow closure and strong harmonics, so $H_2$ approaches $H_1$ in amplitude (low
$H_1$-$H_2$). Light adduction (M2) produces a dominant fundamental with weak harmonics
(high $H_1$-$H_2$). A known limitation is that in the upper register, $H_1$
may coincide with the first vocal tract resonance ($F_1$), contaminating the
measurement.\cite{kreiman2012}

\emph{Spectral Tilt}\cite{degottex2011,drugman2019}: the slope of the linear regression
fitted to the power spectrum (log-frequency vs.\ amplitude in dB) in the 50--5,000~Hz
range. Spectral tilt measures the rate at which vocal energy decays in high
frequencies. In M1, rapid glottal closure creates discontinuities in airflow that
generate strong high-frequency harmonics (shallow tilt, near zero). In M2, smooth
closure produces fewer overtones (steep tilt, very negative). Spectral tilt captures
the global spectral pattern without depending on individual harmonic detection,
making it more robust than $H_1$-$H_2$ in the upper register.

\emph{CPPS per frame}\cite{maryn2015}: smoothed cepstral peak prominence computed per
frame, measuring vibration periodicity. High CPPS indicates clean, periodic phonation
(present in both dense M1 and reinforced M2), while low CPPS indicates noise or
aperiodicity.

\paragraph{Articulatory agility features.}
Pitch velocity ($\Delta f_0/\Delta t$, Hz/s), pitch acceleration
($\Delta^2 f_0/\Delta t^2$, Hz/s$^2$), and syllabic rate (energy peaks/s with
100~ms minimum distance) were computed as indicators of ornamental complexity.

\subsection{Classification Pipeline}
\label{sec:classification}

We implemented four complementary classification methods to ensure robustness and
interpretability:

\paragraph{Method 1: Heuristic threshold.}
A baseline classification using 400~Hz ($\approx$ G4), the empirical
\emph{passaggio} threshold for female voices: frames with $f_0 < 400$~Hz are
classified as M1; frames with $f_0 \geq 400$~Hz as M2.

\paragraph{Method 2: Gaussian Mixture Model (GMM).}
An unsupervised approach using a two-component GMM fitted to the $f_0 \times$ HNR
feature space. Features were normalized using RobustScaler (median/IQR), which is
more resistant to outliers than StandardScaler in degraded recordings. The model was
initialized with \texttt{random\_state=42}.

\paragraph{Method 3: XGBoost with pseudo-labels.}
A supervised classifier\cite{chen2016} trained on GMM-generated pseudo-labels, an
established technique in semi-supervised learning.\cite{lee2013,nigam2000} The model
used nine features ($f_0$, HNR, RMS energy, pitch velocity, pitch acceleration, $F_1$--$F_4$)
with hyperparameters: \texttt{n\_estimators=100}, \texttt{max\_depth=5},
\texttt{learning\_rate=0.1}. Data were split 80\%/20\% for training and testing
(\texttt{random\_state=42}).

We acknowledge that results are reported for a single random seed. Future work
should evaluate performance across multiple seeds and report mean $\pm$ standard
deviation. Additionally, as XGBoost is
trained on GMM outputs rather than ground-truth labels, the reported metrics reflect
cluster separability rather than physiological classification accuracy.

\paragraph{Method 4: Vocal Mechanism Index (VMI).}
A continuous spectral metric (0--1) that replaces the arbitrary frequency threshold
with analysis based on physiologically grounded spectral features. The VMI combines
four normalized components---Alpha Ratio (glottal closure pattern), $H_1$-$H_2$ (adduction
pattern), Spectral Tilt (spectral energy decay rate), and CPPS (vibration
periodicity)---into a single score mapped to five categories: M1\_HEAVY (0.0--0.2),
M1\_LIGHT (0.2--0.4), MIX\_PASSAGGIO (0.4--0.6), M2\_REINFORCED (0.6--0.8), and
M2\_LIGHT (0.8--1.0). The VMI's principal advantage is tessitura independence: it
classifies mechanisms based on spectral signatures rather than absolute frequency
thresholds, making it applicable across voice types and genres.

\subsection{Reproducibility}
\label{sec:reproducibility}

The complete analysis pipeline is implemented in Python 3.10+ and is publicly
available as an open-source repository.\footnote{\url{https://github.com/arthurcornelio88/vocal-analysis}} Several measures ensure reproducibility in
accordance with current best practices for machine learning in voice
research:

\begin{enumerate}
  \item \textbf{Deterministic computation:} All stochastic processes use fixed random
    seeds (\texttt{random\_state=42}), including GMM initialization, XGBoost training,
    and train/test splitting.
  \item \textbf{Locked dependencies:} All library versions are specified in
    \texttt{pyproject.toml} (torchcrepe for pitch extraction, Parselmouth for
    Praat-based features, XGBoost for classification, torchaudio with HTDemucs for
    source separation).
  \item \textbf{Documented parameters:} Every extraction and classification parameter
    is documented with its acoustic justification (e.g., hop length of 220 samples =
    5~ms chosen to capture rapid Choro ornaments; \texttt{weighted\_argmax} decoder
    chosen over \texttt{viterbi} to preserve real high-pitched M2 notes).
  \item \textbf{Cached intermediates:} Source-separated audio, feature matrices, and
    classification outputs are cached at each pipeline stage, enabling independent
    verification of each processing step.
  \item \textbf{Modular architecture:} The pipeline follows a modular design (source
    separation $\rightarrow$ feature extraction $\rightarrow$ mechanism
    classification), allowing individual components to be replaced or updated
    without affecting the overall structure.
\end{enumerate}

Processed feature matrices (CSV) and analysis outputs are available in the project
repository.

%% ============================================================
\section{Results}
\label{sec:results}
%% ============================================================

\subsection{Global Bioacoustic Profile}
\label{sec:global_profile}

Table~\ref{tab:global} presents the global bioacoustic statistics for Ademilde
Fonseca across all three recordings. The mean $f_0$ of 395.3~Hz (G4) places the
singer's central tendency precisely at the theoretical M1/M2 \emph{passaggio}
boundary for female voices. The total range spans from F3 (179.3~Hz) to G5
(781.2~Hz), approximately 2.1 octaves. A mean HNR of 16.9~dB indicates relatively
clean vocal production despite the age and quality of the recordings---a result
attributable to the HTDemucs source separation, which substantially reduced
instrumental interference.

\begin{table}[htbp]
\centering
\caption{Global bioacoustic statistics for Ademilde Fonseca (N = 20,266 voiced frames).}
\label{tab:global}
\begin{tabular}{@{}lrrl@{}}
\toprule
\textbf{Metric} & \textbf{Value} & \textbf{SD} & \textbf{Musical note} \\
\midrule
Mean $f_0$         & 395.3 Hz  & 110.6 Hz & G4  \\
Min $f_0$          & 179.3 Hz  & ---      & F3  \\
Max $f_0$          & 781.2 Hz  & ---      & G5  \\
Mean HNR        & 16.9 dB   & ---      & --- \\
\midrule
\multicolumn{4}{@{}l}{\textit{Per song}} \\
\midrule
Apanhei-te Cavaquinho & 370.1 Hz & --- & F\#4 \\
Delicado              & 382.9 Hz & --- & G4   \\
Brasileirinho         & 444.6 Hz & --- & A4   \\
\bottomrule
\end{tabular}
\end{table}

Per-song analysis reveals systematic variation: \emph{Brasileirinho}, the most
virtuosic piece, has the highest mean $f_0$ (444.6~Hz, A4), reflecting greater use of
the upper register. \emph{Delicado}, a slower piece, concentrates closer to the
comfort tessitura (382.9~Hz, G4). Source separation validation is shown in
Figure~\ref{fig:separation}.

\begin{figure}[htbp]
\centering
\includegraphics[width=\linewidth]{../outputs/plots/apanheite_cavaquinho_separation_validation.png}
\caption{Source separation validation for \emph{Apanhei-te Cavaquinho}. The isolated
vocal track (right) shows elimination of spurious pitch detections in the
instrumental range ($< 200$~Hz) compared to the original mix (left), with a more
continuous melodic contour in the expected vocal range.}
\label{fig:separation}
\end{figure}

\subsection{Mechanism Distribution}
\label{sec:mechanism_distribution}

The threshold-based classification (400~Hz) revealed that 57.3\% of voiced frames
(11,617 frames) operated in M1 (chest voice) and 42.7\% (8,649 frames) in M2 (head
voice), as shown in Table~\ref{tab:mechanism}. The GMM analysis independently
converged on a boundary at approximately 400~Hz, confirming the heuristic threshold.
The overlap between the two clusters in HNR space (both mechanisms showing similar
HNR values of $\approx 17$~dB) indicates that Fonseca maintains consistent vocal
quality across registers (Figure~\ref{fig:mechanism}).

\begin{table}[htbp]
\centering
\caption{Laryngeal mechanism distribution based on 400~Hz threshold
(N = 20,266 frames).}
\label{tab:mechanism}
\begin{tabular}{@{}lrrrrr@{}}
\toprule
\textbf{Mechanism} & \textbf{Frames} & \textbf{\%} & \textbf{Mean $f_0$ (Hz)}
  & \textbf{SD (Hz)} & \textbf{Mean HNR (dB)} \\
\midrule
M1 (chest) & 11,617 & 57.3 & 315.9 & 48.0 & 16.9 \\
M2 (head)  & 8,649  & 42.7 & 502.1 & 75.4 & 16.8 \\
\bottomrule
\end{tabular}
\end{table}

\begin{figure}[htbp]
\centering
\includegraphics[width=\linewidth]{../outputs/plots/mechanism_analysis.png}
\caption{Laryngeal mechanism analysis: (a) $f_0$ histogram by mechanism showing clear
bimodal distribution; (b) $f_0 \times$ HNR scatter plot; (c) HNR boxplot by
mechanism; (d) temporal $f_0$ contour colored by mechanism classification. The bimodal
distribution at 400~Hz confirms the M1/M2 boundary.}
\label{fig:mechanism}
\end{figure}

The M1 distribution is centered at 265--365~Hz (approximately C\#4--F\#4), while M2
occupies 425--580~Hz (approximately A4--D5). The $f_0$ histogram reveals a clear
bimodal distribution with a natural separation at approximately 400~Hz. This value
falls within the M1/M2 overlap zone documented by Roubeau et al.,\cite{roubeau2009}
who reported mean M1$\to$M2 transition frequencies of approximately 312~Hz (E$\flat$4)
for female voices, with the overlap zone spanning roughly one octave---confirming
that 400~Hz lies in the upper portion of the \emph{passaggio} region rather than
at a discrete boundary. Notably, HNR values are nearly
identical between mechanisms (M1: 16.9~dB, M2: 16.8~dB), indicating that the source
separation effectively removed the instrumental noise that would otherwise reduce HNR
in the original recordings.

The temporal contour analysis (Figure~\ref{fig:mechanism}d) reveals that M1/M2
transitions are not random but follow systematic patterns related to musical
structure. Transitions occur predominantly at: (1) phrase endings, where tonal
relaxation coincides with M2 elevation and softer dynamics; (2) ascending ornamental
passages, where rapid glissandi and appoggiaturas traverse the \emph{passaggio}
within a few hundred milliseconds; and (3) sustained notes above G4, where stable M2
maintenance is required for the melodic line. The temporal regularity of these
transitions---visible as consistent blue-to-coral alternations in the $f_0$
contour---suggests a highly controlled vocal technique rather than involuntary
register breaks.

\subsection{XGBoost Classification}
\label{sec:xgboost}

The XGBoost classifier, trained on GMM pseudo-labels with nine features, achieved
perfect classification on the held-out test set (Table~\ref{tab:xgboost}). The
temporal prediction (Figure~\ref{fig:xgboost}) shows coherent mechanism assignments
across all three recordings, with M1 (blue) concentrated in the lower melodic
passages and M2 (coral) in ornamental peaks and phrase endings.

\begin{table}[htbp]
\centering
\caption{XGBoost classification report on held-out test set (20\% of data,
N = 4,054 frames). Features: $f_0$, HNR, energy, $f_0$ velocity, $f_0$ acceleration,
$F_1$--$F_4$. Training labels: GMM clusters.}
\label{tab:xgboost}
\begin{tabular}{@{}lcccc@{}}
\toprule
\textbf{Class} & \textbf{Precision} & \textbf{Recall} & \textbf{F1-Score}
  & \textbf{Support} \\
\midrule
M1 (0)     & 1.00 & 1.00 & 1.00 & 2,492 \\
M2 (1)     & 1.00 & 1.00 & 1.00 & 1,562 \\
\midrule
Accuracy   & \multicolumn{3}{c}{1.00} & 4,054 \\
Macro avg  & 1.00 & 1.00 & 1.00 & 4,054 \\
\bottomrule
\end{tabular}
\end{table}

\begin{figure}[htbp]
\centering
\includegraphics[width=\linewidth]{../outputs/plots/xgb_mechanism_timeline.png}
\caption{Temporal $f_0$ contour colored by XGBoost mechanism prediction for each
analyzed recording. Blue: M1 (chest voice); coral: M2 (head voice). Note the fluid
alternation between mechanisms, particularly in ornamental passages.}
\label{fig:xgboost}
\end{figure}

The perfect accuracy must be interpreted with caution. Because XGBoost was trained on
GMM-generated pseudo-labels rather than ground-truth annotations, this result
reflects the high separability of the GMM clusters in the nine-dimensional feature
space---not validated physiological classification accuracy. The high concordance
between GMM and XGBoost nevertheless suggests that Fonseca's laryngeal mechanisms
follow consistent, predictable patterns characteristic of consolidated vocal
technique.

\subsection{Vocal Mechanism Index (VMI)}
\label{sec:vmi_results}

The VMI analysis (Table~\ref{tab:vmi}) reveals a distribution dominated by the
MIX\_PASSAGGIO category (44.3\% of frames), indicating that nearly half of Fonseca's
vocal production operates in the transitional zone between mechanisms. M1\_LIGHT
constitutes 38.6\%, reflecting the predominance of thin-edge chest voice in the
middle register. Together, these two categories account for 82.9\% of all frames,
demonstrating that the singer's technique centers on the mixing zone rather than
extreme mechanism deployment.

\begin{table}[htbp]
\centering
\caption{VMI category distribution across all recordings (N = 20,266 frames).}
\label{tab:vmi}
\begin{tabular}{@{}lrrccc@{}}
\toprule
\textbf{Category} & \textbf{Frames} & \textbf{\%} & \textbf{Mean VMI}
  & \textbf{Alpha Ratio (dB)} & \textbf{$H_1$-$H_2$ (dB)} \\
\midrule
M1\_HEAVY       & 995   & 4.9  & 0.154 & 1.8   & $-20.9$ \\
M1\_LIGHT       & 7,828 & 38.6 & 0.313 & $-5.0$  & $-3.9$  \\
MIX\_PASSAGGIO  & 8,977 & 44.3 & 0.492 & $-13.0$ & 12.2    \\
M2\_REINFORCED  & 2,331 & 11.5 & 0.664 & $-19.9$ & 29.5    \\
M2\_LIGHT       & 135   & 0.7  & 0.851 & $-33.3$ & 46.4    \\
\bottomrule
\end{tabular}
\end{table}

The spectral features show clear progression across categories: Alpha Ratio
decreases from 1.8~dB (M1\_HEAVY) to $-33.3$~dB (M2\_LIGHT), reflecting the shift
from strong upper harmonics (firm glottal closure in M1) to energy concentrated in
the fundamental (gentle closure in M2). $H_1$-$H_2$ increases from $-20.9$~dB to
46.4~dB, consistent with the transition from firm adduction (low $H_1$-$H_2$) to light
adduction (high $H_1$-$H_2$).

The near-absence of M2\_LIGHT (0.7\% of frames) is notable: Fonseca rarely employs
pure falsetto, preferring instead the reinforced M2 or \emph{voix mixte} quality
(11.5\%) that maintains glottal adduction and frontal resonance while operating in
M2. This finding aligns with the aesthetic demands of Choro, where vocal
projection---even in M2---must compete with dense instrumental arrangements.

Per-song VMI analysis reveals systematic variation correlated with musical character.
\emph{Brasileirinho} (mean VMI = 0.449), the most virtuosic piece requiring rapid
ornamental passages in the upper register, shows the highest passaggio engagement.
\emph{Apanhei-te Cavaquinho} (VMI = 0.431), with its wide amplitude range including
G5 peaks, also centers in the mixing zone. \emph{Delicado} (VMI = 0.413), a slower
piece with moderate tempo, shows the strongest M1\_LIGHT predominance (44.0\%),
reflecting the singer's preference for the comfort tessitura in less technically
demanding passages. This stylistic modulation of mechanism deployment---adapting
the M1/M2 balance to the musical context rather than conforming to a fixed vocal
type---is precisely the biomechanical flexibility that the \emph{Fach} system
cannot accommodate.

The VMI scatter plot (Figure~\ref{fig:vmi}) illustrates the continuous nature of
mechanism transitions, in contrast to the binary threshold approach. The gradual
color gradient in the 350--450~Hz range demonstrates that the \emph{passaggio} is
not a discrete boundary but a continuous zone of spectral transformation.

\begin{figure}[htbp]
\centering
\includegraphics[width=\linewidth]{../outputs/plots/vmi_scatter.png}
\caption{VMI scatter plot: $f_0$ versus Alpha Ratio colored by VMI value (0--1,
RdBu\_r scale). Low VMI (red) indicates M1; high VMI (blue) indicates M2. The
continuous color gradient demonstrates fluid mechanism transitions, particularly
in the 350--450~Hz range.}
\label{fig:vmi}
\end{figure}

\begin{figure}[htbp]
\centering
\includegraphics[width=\linewidth]{../outputs/plots/excerpt_brasileirinho.png}
\caption{Five-second excerpt from \emph{Brasileirinho} at the densest $f_0$ window.
Left axis: frequency (Hz); right axis: musical notes. Colored by mechanism
prediction. Note the rapid M1/M2 alternations in ornamental passages, demonstrating
the singer's articulatory agility.}
\label{fig:excerpt}
\end{figure}

\subsection{Formant Analysis and Diction Preservation}
\label{sec:formants}

Formant extraction revealed speech-like articulatory configurations across all
recordings: mean $F_1$ = 660.1~Hz, mean $F_2$ = 1,629~Hz, mean $F_3$ = 2,798~Hz, and mean
$F_4$ = 3,802~Hz. The $F_1$ and $F_2$ values approach those of colloquial Brazilian
Portuguese open vowels, suggesting that Fonseca prioritizes natural language
articulation rather than the ``vowel darkening'' typical of classical singing, where
$F_1$ is systematically lowered to achieve a more ``covered'' timbre. This finding is
consistent with Tatit's concept of \emph{figurativiza\c{c}\~{a}o}---the
primacy of speech-like intonation in Brazilian popular song.\cite{tatit2002}

The preservation of speech-like formant patterns at high articulatory speeds is
particularly noteworthy. Even in the rapid ornamental passages of
\emph{Brasileirinho}, where syllabic rates exceed 5 syllables per second, the
formant structure maintains its speech-like configuration. This suggests a vocal
technique specifically optimized for textual intelligibility---a competency that
Behlau and Ziemer\cite{behlau1988} identify as essential for Brazilian popular
singing but that has no equivalent criterion in the \emph{Fach} system.

The stable $F_3$ and $F_4$ ranges (2,798~Hz and 3,802~Hz) indicate some degree of
singer's formant cluster presence, which contributes to vocal projection in the dense
instrumental textures of Choro. However, unlike operatic singing where the singer's
formant is a defining feature of \emph{Fach} category, in Fonseca's case it appears
as a secondary acoustic byproduct of efficient vocal production rather than a primary
technical objective.

%% ============================================================
\section{Discussion}
\label{sec:discussion}
%% ============================================================

\subsection{Implications for Fach Classification}
\label{sec:fach_implications}

The computational analysis of Ademilde Fonseca's vocal production provides empirical
evidence against the universal applicability of the \emph{Fach} system. Three key
findings support this conclusion.

First, the singer integrates both laryngeal mechanisms fluidly, with 57.3\% M1 and
42.7\% M2 utilization across her repertoire. This near-equal distribution, combined
with 44.3\% of frames in the VMI passaggio zone, demonstrates a vocal technique
built on continuous mechanism modulation rather than the discrete register categories
assumed by \emph{Fach} classification. Attempting to assign a \emph{Fach} category
would require privileging one mechanism over the other, fundamentally
mischaracterizing the biomechanical reality.

Second, formant analysis reveals speech-like articulation patterns ($F_1 \approx$
660~Hz, $F_2 \approx$ 1,629~Hz) consistent with open vowel configurations in
Brazilian Portuguese. This finding supports Tatit's concept of
\emph{figurativiza\c{c}\~{a}o}\cite{tatit2002}: the singer prioritizes natural language
articulation, avoiding the vowel ``darkening'' typical of operatic singing, where $F_1$
is lowered to achieve a more ``covered'' timbre. The stable $F_3$ and $F_4$ values
(2,798~Hz and 3,802~Hz) suggest some singer's formant presence, contributing to
projection in dense arrangements, but the overall spectral profile is
speech-dominated rather than operatically optimized.

Third, the mechanism distribution varies systematically with musical demand.
\emph{Delicado} (moderate tempo, mean $f_0$ = 382.9~Hz) concentrates in the comfort
tessitura with M1 predominance (M1\_LIGHT: 44.0\%), while \emph{Brasileirinho}
(virtuosic, mean $f_0$ = 444.6~Hz) shows greater M2 proportion and higher VMI (0.449),
reflecting the adaptation of mechanism deployment to musical context. Notably,
\emph{Brasileirinho} exhibits the highest MIX\_PASSAGGIO proportion (51.0\%),
indicating that over half of the vocal production in this virtuosic piece operates in
the transitional zone---a direct consequence of the rapid ornamental passages that
demand fluid mechanism alternation. This stylistic adaptability---adjusting the
M1/M2 balance to serve the music rather than conforming to a fixed
category---is precisely what the \emph{Fach} system cannot capture.

The visual analysis of musical excerpts (Figure~\ref{fig:excerpt}) confirms these
findings at the micro-structural level. M1/M2 transitions occur predominantly at
three structural positions: (1) phrase endings, where tonal relaxation coincides with
M2 elevation; (2) ascending ornamental passages, where rapid glissandi and
appoggiaturas traverse the \emph{passaggio}; and (3) sustained notes above G4, where
stable M2 maintenance is required. The high concordance between GMM and XGBoost
predictions for these transitions suggests that Fonseca's laryngeal mechanisms follow
consistent, predictable patterns---characteristic of consolidated vocal technique
developed over decades of professional performance.

\subsection{VMI as a Portable Classification Metric}
\label{sec:vmi_discussion}

The Vocal Mechanism Index demonstrates several advantages over traditional
classification approaches. Unlike the heuristic threshold (400~Hz), VMI operates on
spectral features (Alpha Ratio, $H_1$-$H_2$, Spectral Tilt, CPPS) that reflect
physiological properties of glottal behavior rather than absolute frequency values.
This makes VMI inherently tessitura-agnostic: the same spectral signatures indicate
M1 or M2 regardless of whether the singer is a bass ($f_0 \approx 100$~Hz) or a
soprano ($f_0 \approx 600$~Hz).

The five-category VMI scale captures gradations lost in binary M1/M2 classification.
The distinction between M1\_HEAVY (dense chest voice, 4.9\% of frames) and M1\_LIGHT
(thin-edge chest voice, 38.6\%) provides clinically and pedagogically relevant
information about the degree of vocal fold adduction. Similarly, separating
M2\_REINFORCED (11.5\%) from M2\_LIGHT (0.7\%) distinguishes the \emph{voix mixte}
technique from pure falsetto---a distinction with clear implications for vocal
pedagogy and health assessment.

Although developed for Choro, the VMI pipeline is genre-agnostic by design. The
modular architecture (source separation $\rightarrow$ feature extraction
$\rightarrow$ mechanism classification) can be applied to any solo vocal repertoire
with parameter adjustments: \texttt{fmin}/\texttt{fmax} for tessitura range (e.g.,
50--400~Hz for operatic bass, 200--1200~Hz for soprano), HTDemucs for accompanied
vs.\ \emph{a cappella} recordings, and the VMI threshold boundaries for
tessitura-specific \emph{passaggio} zones.\cite{bozeman2013,sundberg1987}
The spectral weights used in VMI (Alpha Ratio, $H_1$-$H_2$, Spectral Tilt) are
tessitura-independent by construction, as they measure relative energy distribution
and glottal behavior patterns rather than absolute frequency values. The same
spectral tilt indicating firm M1 closure in a bass at 120~Hz applies equally to a
soprano at 500~Hz---the underlying physics of glottal closure is mechanism-dependent,
not frequency-dependent.\cite{degottex2011,drugman2019}

\subsection{Limitations}
\label{sec:limitations}

This study has several limitations that should be considered when interpreting the
results.

\paragraph{Historical recording quality.}
The analyzed recordings date from the 1940s--1960s and exhibit low SNR, elevated
background noise, and spectral degradation. While HTDemucs source separation
substantially improved the signal quality, spectral features (particularly CPPS and
$H_1$-$H_2$) may be less reliable than in modern, studio-quality recordings. The
post-separation HNR of $\approx 17$~dB, while adequate, is lower than the
$> 20$~dB typically expected for reliable acoustic voice analysis.

\paragraph{Pseudo-label circularity.}
The XGBoost classifier was trained on labels generated by the GMM, creating a
circular dependency. The perfect classification accuracy ($F_1$ = 1.00) reflects the
high separability of GMM clusters in the nine-dimensional feature space, not
validated physiological accuracy. XGBoost cannot, on average, be more accurate than
its pseudo-labels, and inherits any systematic errors from the GMM (confirmation
bias).\cite{lee2013}

\paragraph{Gaussian assumption.}
The GMM assumes that M1 and M2 clusters follow Gaussian distributions in the
$f_0 \times$ HNR space. Real mechanism distributions---particularly at the
\emph{passaggio}, where biomechanical instability produces non-linear
behavior---may deviate substantially from this assumption.

\paragraph{Absence of ground truth.}
No electroglottographic (EGG) or laryngoscopic validation was available for these
historical recordings. Without such ground truth, the true classification accuracy
remains unknown. The convergence of threshold, GMM, XGBoost, and VMI methods provides
internal consistency, but not external validation.

\paragraph{Small corpus.}
The analysis is based on three recordings of a single singer. While Ademilde Fonseca
is artistically representative of Choro singing, generalization to other
singers---with different vocal techniques, tessituras, and recording
conditions---requires an expanded dataset encompassing multiple artists and
recording eras.

\paragraph{Single random seed.}
Results for the GMM and XGBoost models are reported for a single random seed
(\texttt{random\_state=42}). Future work should evaluate model stability across
multiple seeds and report mean $\pm$ standard deviation of evaluation metrics.

\paragraph{VMI fixed weights.}
The current VMI implementation uses heuristic weights for its four spectral
components rather than empirically optimized values. Training these weights via
supervised learning with EGG-validated labels is a priority for future development.

\paragraph{$H_1$-$H_2$ at high $f_0$.}
In the upper register, the first harmonic ($H_1$) may coincide with the
first vocal tract resonance ($F_1$), contaminating the $H_1$-$H_2$
measurement.\cite{kreiman2012} Spectral Tilt is included as a complementary feature
to mitigate this limitation, as it captures the global spectral energy distribution
without depending on individual harmonic detection.

\paragraph{Breathiness in M1.}
Low HNR and CPPS values are treated as probabilistic indicators of M2, but breathy
phonation can also occur in M1 (e.g., intentional \emph{voix souffl\'{e}e}). These
features are therefore not deterministic discriminators but statistical tendencies.

%% ============================================================
\section{Conclusion}
\label{sec:conclusion}
%% ============================================================

This study provides the first computational bioacoustic analysis of laryngeal
mechanisms in Brazilian Choro singing, using the historical recordings of Ademilde
Fonseca as a case study. Through a hybrid pipeline combining neural source
separation (HTDemucs), CNN-based pitch tracking (CREPE), spectral feature extraction
(Praat/Parselmouth), and multi-method classification (threshold, GMM, XGBoost, VMI),
we demonstrate that the singer employs both M1 and M2 mechanisms fluidly, with
nearly half of her vocal production operating in the mixing/\emph{passaggio} zone.

The empirical evidence challenges the universal applicability of the \emph{Fach}
system: Ademilde Fonseca's vocal technique prioritizes colloquial expressivity,
textual intelligibility, and rhythmic agility---vocal qualities that the \emph{Fach}
framework, designed for operatic acoustic projection, cannot adequately classify.
Her voice does not fit operatic categories because it operates in a fundamentally
different vocal paradigm.

We contribute: (1) an open-source, reproducible pipeline specifically designed for
degraded historical recordings, adaptable to other genres and tessituras; (2) the
Vocal Mechanism Index (VMI), a continuous spectral metric for tessitura-agnostic
mechanism classification; and (3) quantitative evidence supporting the need for
genre-specific vocal classification frameworks grounded in laryngeal physiology
rather than operatic aesthetics.

Several directions for future work emerge from these findings. First, the corpus
should be expanded to include multiple Choro singers across different eras, voice
types, and recording conditions, enabling statistical generalization beyond a single
artist. Second, VMI should be validated against electroglottographic (EGG)
measurements, which provide direct physiological ground truth for laryngeal mechanism
classification.\cite{henrich2004} Third, the VMI weights---currently
heuristic---should be optimized via supervised learning using EGG-validated labels,
potentially through the evolutionary parameter tuning approach demonstrated by
Boratto et al.\cite{boratto2025} Fourth, model stability should be evaluated across
multiple random seeds, with mean $\pm$ standard deviation of classification metrics
reported. Fifth, the
pipeline should be applied to other microphone-dependent genres---MPB, fado, vocal
jazz, sacred music---to assess the VMI's cross-genre portability.

The ultimate goal is to establish a vocological framework in which vocal identity is
defined by biomechanical behavior---not by the categories of a classification system
designed for a single genre, in a single cultural tradition, over a century ago.

%% ============================================================
\section*{Declaration of generative AI and AI-assisted technologies in the
manuscript preparation process}
%% ============================================================

During the preparation of this work the author used Google Gemini 2.0 Flash in order
to generate a preliminary narrative interpretation of bioacoustic analysis plots and
descriptive statistics. The author also used Anthropic Claude Code for code
development assistance in the computational analysis pipeline and manuscript
preparation. After using these tools, the author reviewed and edited all content as
needed and takes full responsibility for the content of the publication.

%% ============================================================
\section*{Declaration of competing interests}
%% ============================================================

The author declares no competing interests.

%% ============================================================
\section*{Funding}
%% ============================================================

This research did not receive any specific grant from funding agencies in the
public, commercial, or not-for-profit sectors.

%% ============================================================
\section*{Data availability}
%% ============================================================

The analysis code and processed feature data are publicly available at
\url{https://github.com/arthurcornelio88/vocal-analysis}. Raw audio recordings are
subject to copyright and cannot be redistributed.

%% ============================================================
\bibliographystyle{elsarticle-num}
\bibliography{references}

\end{document}
