\documentclass[12pt]{letter}

\usepackage[utf8]{inputenc}
\usepackage[T1]{fontenc}
\usepackage[margin=1in]{geometry}
\usepackage{hyperref}

\signature{Arthur Cornelio}
\address{Rio de Janeiro, Brazil \\
  \href{mailto:arthur.cornelio@gmail.com}{arthur.cornelio@gmail.com}}

\begin{document}

\begin{letter}{Editor-in-Chief \\
  Journal of Voice \\
  The Voice Foundation}

\opening{Dear Editor,}

I am writing to submit our manuscript entitled \textbf{``Multidimensional
Vocological Analysis: Laryngeal Mechanism Physiology, Digital Signal Processing,
and a Critique of the Fach System in Brazilian Choro''} for consideration as an
original research article in the \textit{Journal of Voice}.

This study presents the first computational bioacoustic analysis of laryngeal
mechanisms (M1/M2) in Brazilian Choro singing, using historical recordings of
Ademilde Fonseca as a case study. Our work makes three contributions to the field:

\begin{enumerate}
  \item \textbf{An open-source, reproducible pipeline} combining neural source
    separation (HTDemucs), CNN pitch tracking (CREPE), spectral feature extraction
    (Praat/Parselmouth), and multi-method classification (GMM, XGBoost)---designed
    specifically for degraded historical recordings and adaptable to other genres.

  \item \textbf{The Vocal Mechanism Index (VMI)}, a continuous spectral metric
    (0--1) for tessitura-agnostic laryngeal mechanism classification based on
    Alpha Ratio, H1-H2, Spectral Tilt, and CPPS, offering a physiologically
    grounded alternative to fixed frequency thresholds.

  \item \textbf{Empirical evidence challenging the universal applicability of the
    Fach system}, demonstrating that Choro vocal technique prioritizes colloquial
    expressivity and articulatory agility over the acoustic projection criteria
    central to operatic voice classification.
\end{enumerate}

We believe this manuscript aligns well with the \textit{Journal of Voice}'s scope
in voice medicine and voice science, particularly in its intersection of laryngeal
physiology, machine learning, and vocal classification. The findings are relevant
to voice researchers, speech-language pathologists, and vocal pedagogues working
with non-Western and popular music genres.

This manuscript has not been previously published and is not under consideration
elsewhere.

\closing{Sincerely,}

\end{letter}

\end{document}
